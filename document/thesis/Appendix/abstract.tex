\chapter*{Tóm tắt}
\label{abstract}

Đứng trước sự bùng nổ dữ liệu tại thiết bị biên, các phương pháp máy học truyền thống (đòi hỏi việc truyền dữ liệu từ các thiết bị cuối về một máy chủ mạnh mẽ để huấn luyện) bộc lộ nhiều nhược điểm về chi phí phần cứng và vấn đề quyền riêng tư dữ liệu của người dùng. Mặt khác, khả năng lưu trữ và tính toán tại các thiết bị biên đang ngày càng được cải thiện và nâng cao cũng như phát sự triển vượt bậc của máy học trong nhiều lĩnh vực. Thực tế này thúc đẩy việc nghiên cứu một phương pháp học tối ưu hơn về chi phí và đảm bảo quyền riêng tư dữ liệu người dùng.

Khái niệm \textit{Federated Learning} (FL) cùng thuật toán \textit{Federated Averaging} (\codeword{FedAvg}) ra đời, được xem như một giải pháp thay thế, rất phù hợp với thực tế nêu trên \cite{mcmahan2017communication}. Giải pháp này không những đạt hiệu quả gần như tương đương các phương pháp học sâu đã có, giải quyết được vấn đề chi phí phần cứng, mà còn đảm bảo được quyền riêng tư dữ liệu người dùng. Tuy nhiên, đứng trước dữ liệu không đồng nhất và có tính cá nhân hóa cao trên từng người dùng (dữ liệu Non-IID), hệ thống này bị suy giảm hiệu suất nghiêm trọng \cite{zhao2018federated}.

Khoá luận này nhằm mục đích khảo sát khái niệm \textit{Federated Learning} và vấn đề tối ưu hệ thống FL trên dữ liệu Non-IID. Bằng cách kết hợp các thuật toán huấn luyện \textit{Meta-Learning} \cite{hospedales2020meta} (ML) và sử dụng kỹ thuật \textit{Personalization Layer} \cite{zhu2021federated} (PL) vào hệ thống FL, \codeword{FedMeta-Per} - thuật toán đề xuất của khoá luận đã đạt hiệu quả cao về độ chính xác và tính cá nhân hóa trên từng người dùng khi so sánh với thuật toán \codeword{FedAvg}, thuật toán \codeword{FedPer} \cite{arivazhagan2019federated} (tối ưu hệ thống FL bằng PL) và các thuật toán \codeword{FedMeta} \cite{chen2018federated} (tối ưu hệ thống FL bằng ML).

\chapter*{Lời cảm ơn}
\label{thanks}

Thời gian làm khóa luận kéo dài 6 tháng, đối với chúng tôi mà nói, là khoảng thời gian học thuật đáng nhớ nhất trong quãng đời sinh viên. Bằng những tri thức tiên nghiệm do GS. TS. Lê Hoài Bắc cung cấp, chúng tôi tiến hành tự tìm hiểu đề tài này trong nhiều bước huấn luyện và kiểm thử bản thân bằng cách tiến hành tổng hợp kiến thức và thực hiện các thí nghiệm liên quan (sẽ được trình bày kỹ trong các phần sau của báo cáo). Do đó, xin phép gửi lời cảm ơn chân thành nhất đến GS. TS. Lê Hoài Bắc, người đã cho chúng tôi cơ hội thử thách chính bản thân bằng một đề tài hết sức thú vị như thế này.

Tất nhiên, một mình chúng tôi không thể tự tin thực hiện hết tất cả các phần trong khóa luận. Rất nhiều lúc, chúng tôi bế tắc trước những câu hỏi vì chính bản thân đề tài này là một lĩnh vực nghiên cứu khá mới nên chưa được cộng đồng hỗ trợ nhiều. Một trong những người mà chúng tôi nghĩ đến lúc đó, xin được phép nhắc tên và cảm ơn chị Bùi Thị Cẩm Nhung, cựu sinh viên lớp cử nhân tài năng, khóa K2017. Không có những lời khuyên từ chị, chúng tôi sẽ không bao giờ biết được mình cần bao nhiêu VRAM để huấn luyện mô hình này cũng như không bao giờ biết được sự tồn tại của một người rất giỏi và nhiệt tình như chị.

Chúng tôi cũng rất muốn gửi một lời cảm ơn đến TS. Nguyễn Tiến Huy, người sẵn sàng giải đáp băn khoăn của chúng tôi trong thời gian 30 phút kể lúc email được gửi đi, cũng như đội ngũ cán bộ giảng dạy của Khoa Công nghệ thông tin, Trường Đại học Khoa học Tự nhiên, những người đã trao cho chúng tôi những tri thức thậm chí còn đáng giá hơn nhiều lần so với mức học phí gần 300.000VND/tín chỉ.

Lời cảm ơn cuối cùng, chúng tôi xin được phép dành riêng cho gia đình, những người đã hỗ trợ chúng tôi cả về vật chất lẫn tinh thần trong không những suốt 4 năm đại học mà còn trong 18 năm đầu đời. Đóng góp của họ là rất lớn và không thể nào đo đạc dựa trên trình độ vật lý hiện tại của chúng tôi.

% Template KLTN cho SV trường ĐHKHTN
% Liên hệ: nqminh@fit.hcmus.edu.vn
% Last update: 08/06/2016

% Chú ý: đọc các phần chú ý đóng khung của file này và chỉnh lại cho phù hợp.
% Trước khi build, xóa hết các file được tạo ra trong quá trình build trước đó, và build theo thứ tự: BIB > PDF > PDF.
% Nếu cập nhật tài liệu tham khảo, cũng cần build lại theo cách trên.

\documentclass[oneside,a4paper,14pt]{extreport}

% Font tiếng Việt
\usepackage[T5]{fontenc}
\usepackage[utf8]{inputenc}
\DeclareTextSymbolDefault{\DH}{T1}

% Tài liệu tham khảo
\usepackage[
	sorting=nty,
	backend=bibtex,
	defernumbers=true]{biblatex}
\usepackage[unicode]{hyperref} % Bookmark tiếng Việt
\addbibresource{References/references.bib}

\makeatletter
\def\blx@maxline{77}
\makeatother

% Chèn hình, các hình trong luận văn được để trong thư mục Images/
\usepackage{graphicx}
\graphicspath{ {Images/} }

% Chèn và định dạng mã nguồn
\usepackage{listings}
\usepackage{color}
\definecolor{codegreen}{rgb}{0,0.6,0}
\definecolor{codegray}{rgb}{0.5,0.5,0.5}
\definecolor{codepurple}{rgb}{0.58,0,0.82}
\definecolor{backcolour}{rgb}{0.95,0.95,0.92}
\lstdefinestyle{mystyle}{
    backgroundcolor=\color{backcolour},   
    commentstyle=\color{codegreen},
    keywordstyle=\color{magenta},
    numberstyle=\tiny\color{codegray},
    stringstyle=\color{codepurple},
    basicstyle=\footnotesize,
    breakatwhitespace=false,         
    breaklines=true,                 
    captionpos=b,                    
    keepspaces=true,                 
    numbers=left,                    
    numbersep=5pt,                  
    showspaces=false,                
    showstringspaces=false,
    showtabs=false,                  
    tabsize=2
}
\lstset{style=mystyle}

% Chèn và định dạng mã giả
\usepackage{amsmath}
\usepackage{algorithm}
\usepackage[noend]{algpseudocode}
\makeatletter
\def\BState{\State\hskip-\ALG@thistlm}
\makeatother

% Bảng biểu
\usepackage{multirow}
\usepackage{array}
\newcolumntype{L}[1]{>{\raggedright\let\newline\\\arraybackslash\hspace{0pt}}m{#1}}
\newcolumntype{C}[1]{>{\centering\let\newline\\\arraybackslash\hspace{0pt}}m{#1}}
\newcolumntype{R}[1]{>{\raggedleft\let\newline\\\arraybackslash\hspace{0pt}}m{#1}}



 
% Đổi tên mặc định
\renewcommand{\chaptername}{Chương}
\renewcommand{\figurename}{Hình}
\renewcommand{\tablename}{Bảng}
\renewcommand{\contentsname}{Mục lục}
\renewcommand{\listfigurename}{Danh sách hình}
\renewcommand{\listtablename}{Danh sách bảng}
\renewcommand{\appendixname}{Phụ lục}

% Định dạng chapter
\usepackage{titlesec}
\titleformat{\chapter}
    [display] 
    {\normalfont\bfseries\Large}{\chaptername \ \thechapter}{10pt}{\huge}
\titlespacing*{\chapter}{0pt}{-10pt}{40pt} %khoảng cách giữa chapter và đầu trang

\titleformat{\section}
    {\normalfont\bfseries\large}{\thesection}{1em}{}
 
\titleformat{\subsection}
    {\normalfont\bfseries\normalsize}{\thesubsection}{1em}{}
  
% Dãn dòng 1.5
\usepackage{setspace}
\onehalfspacing

% Thụt vào đầu dòng
\usepackage{indentfirst}

% Canh lề
\usepackage[
  top=20mm,
  bottom=10mm,
  left=30mm,
  right=20mm,
  footskip = 15mm,
  includefoot]{geometry}
  
% Trang bìa
\usepackage{tikz}
\usetikzlibrary{calc}
\newcommand\HRule{\rule{\textwidth}{1pt}}

% ========================================================================================= %
% CHÚ Ý: Thông tin chung về KLTN - sinh viên điền vào đây để tự động update các trang khác  %
% ========================================================================================= %
\newcommand{\tenSV}{Nguyễn~Văn~A~-~Trần~Văn~B} % Dấu ~ là khoảng trắng không được tách (các chữ nối với nhau bằng dấu ~ sẽ nằm cùng 1 dòng
\newcommand{\mssv}{1234567}
\newcommand{\tenKL}{Sử~dụng~LaTeX trong Khoá~luận~tốt~nghiệp} % Chú ý dấu ~ trong tên khóa luận
\newcommand{\tenGVHD}{Tên~Giáo~Viên}
\newcommand{\tenBM}{Công nghệ tri thức}

\begin{document}

\begin{titlepage}

\begin{center}
%ĐẠI HỌC QUỐC GIA THÀNH PHỐ HỒ CHÍ MINH\\
TRƯỜNG ĐẠI HỌC KHOA HỌC TỰ NHIÊN\\
\textbf{KHOA CÔNG NGHỆ THÔNG TIN}\\[2cm]


{ \Large \bfseries Nguyễn Văn A - Trần Thị B\\[2cm] } 

%Tên đề tài Khóa luận tốt nghiệp/Đồ án tốt nghiệp

{ \Large \bfseries HƯỚNG DẪN TRÌNH BÀY BÁO CÁO KHÓA LUẬN TỐT NGHIỆP, \\ĐỒ ÁN TỐT NGHIỆP, \\THỰC TẬP TỐT NGHIỆP \\[3cm]} 


%Chọn trong các dòng sau
\large KHÓA LUẬN TỐT NGHIỆP CỬ NHÂN\\
%\large ĐỒ ÁN TỐT NGHIỆP CỬ NHÂN\\
%\large THỰC TẬP TỐT NGHIỆP CỬ NHÂN\\
%Đưa vào dòng này nếu thuộc chương trình Chất lượng cao, hoặc lớp Cử nhân tài năng
\large CHƯƠNG TRÌNH CHÍNH QUY\\
%\large CHƯƠNG TRÌNH CHẤT LƯỢNG CAO\\
%\large CHƯƠNG TRÌNH CỬ NHÂN TÀI NĂNG\\[2cm]


\begin{tikzpicture}[remember picture, overlay]
  \draw[line width = 2pt] ($(current page.north west) + (2cm,-2cm)$) rectangle ($(current page.south east) + (-1.5cm,2cm)$);
\end{tikzpicture}

\vfill
Tp. Hồ Chí Minh, tháng MM/YYYY

\end{center}

\pagebreak



\begin{center}

TRƯỜNG ĐẠI HỌC KHOA HỌC TỰ NHIÊN\\
\textbf{KHOA CÔNG NGHỆ THÔNG TIN}\\[2cm]


{\large \bfseries Nguyễn Văn A - 1512090\\} 
{\large \bfseries Trần Thị B - 1512007\\[2cm]}

%Tên đề tài Khóa luận tốt nghiệp/Đồ án tốt nghiệp

{ \Large \bfseries HƯỚNG DẪN TRÌNH BÀY BÁO CÁO KHÓA LUẬN TỐT NGHIỆP, \\ĐỒ ÁN TỐT NGHIỆP,\\THỰC TẬP TỐT NGHIỆP\\[2cm] } 


%Chọn trong các dòng sau
\large KHÓA LUẬN TỐT NGHIỆP CỬ NHÂN\\
%\large ĐỒ ÁN TỐT NGHIỆP CỬ NHÂN\\
%Đưa vào dòng này nếu thuộc chương trình Chất lượng cao, hoặc lớp Cử nhân tài năng
\large CHƯƠNG TRÌNH CHÍNH QUY\\[2cm]
%\large CHƯƠNG TRÌNH CHẤT LƯỢNG CAO\\[2cm]
%\large CHƯƠNG TRÌNH CỬ NHÂN TÀI NĂNG\\[2cm]

\textbf{GIÁO VIÊN HƯỚNG DẪN}\\
PGS.TS. Hoàng Văn C\\
TS. Lê Thị D

\begin{tikzpicture}[remember picture, overlay]
  \draw[line width = 2pt] ($(current page.north west) + (2cm,-2cm)$) rectangle ($(current page.south east) + (-1.5cm,2cm)$);
\end{tikzpicture}

\vfill
Tp. Hồ Chí Minh, tháng MM/YYYY

\end{center}

\end{titlepage}
% Sasu trang Title, các bạn chèn nhận xét gủa GVHD và GVPB. Nhận xét sẽ được giáo vụ phát sau buổi bảo vệ để các bạn đóng quyển.

\pagenumbering{roman} % Đánh số i, ii, iii, ...

%\addcontentsline{toc}{chapter}{Lời cam đoan}
%\chapter*{Lời cam đoan}
\label{reassurances}

Tôi xin cam đoan đây là công trình nghiên cứu của riêng tôi. Các số liệu và kết quả nghiên cứu trong luận văn này là trung thực và không trùng lặp với các đề tài khác.

\addcontentsline{toc}{chapter}{Lời cảm ơn}
\chapter*{Lời cảm ơn}
\label{thanks}

Tôi xin chân thành cảm ơn ...

\addcontentsline{toc}{chapter}{Đề cương chi tiết}
\include{Appendix/decuong}

% Mục lục, danh sách hình, danh sách bảng
\addcontentsline{toc}{chapter}{Mục lục}
\tableofcontents
\listoffigures
\listoftables

\addcontentsline{toc}{chapter}{Tóm tắt}
\include{Appendix/tomtat}

\clearpage

\pagenumbering{arabic} % Đánh số 1, 2, 3, ...

% Các chương nội dung
\chapter{Giới thiệu}
\label{Chapter1}

\section{Đặt vấn đề \& Động lực}

Hiện nay, các thiết bị biên như điện thoại, máy tính bảng, thậm chí máy giặt, máy hút bụt thông minh có thể sinh ra lượng lớn dữ liệu trong quá trình hoạt động. Lượng dữ liệu này, nếu tận dụng được, có thể mang lại sự cải thiện rất lớn về độ chính xác cho các mô hình máy học hiện tại. Ví dụ, dữ liệu thu thập được từ bàn phím điện thoại có thể phục vụ tối ưu cho các mô hình ngôn ngữ; ảnh chụp được lưu trữ trong bộ nhớ điện thoại hoàn toàn có thể được sử dụng làm dữ liệu để huấn luyện cho mô hình nhận dạng ảnh; hay lịch sử duyệt web của người dùng có thể được dùng cho bài toán đề xuất sản phẩm. Những lý do trên trở thành một động lực to lớn, thúc đẩy việc tìm ra một phương pháp giúp tận dụng nguồn dữ liệu dồi dào này.

Việc ngày càng nhiều dữ liệu được sinh ra tại các thiết bị biên khiến cho phương pháp huấn luyện mô hình theo cách tiếp cận truyền thống (được gọi là huấn luyện tập trung) bộc lộ nhiều khuyết điểm. Ba điểm yếu khiến cho cách tiếp cận này không còn mạnh mẽ có thể kể đến: (1) - Sự vi phạm về quyền riêng tư dữ liệu, (2) - Chi phí truyền tin, (3) - Chi phí phần cứng máy chủ.

\textbf{Sự vi phạm về quyền riêng tư dữ liệu.} Phương pháp truyền thống đòi hỏi phải gửi dữ liệu người dùng về một máy chủ để tiến hành huấn luyện mô hình. Các thông tin nhạy cảm của người dùng hoàn toàn có thể bị nghe lén bởi kẻ tấn công hoặc bị khai thác khi máy chủ bị nhiễm mã độc. Điều này ảnh hưởng nghiêm trọng đến quyền riêng tư dữ liệu của người dùng - một vấn đề mà hiện nay đang nhận được rất nhiều sự quan tâm từ cả người dùng lẫn chính phủ.

\textbf{Chi phí truyền tin.} Dữ liệu sinh ra tại thiết bị biên đang ngày một tăng lên do văn hóa sử dụng và sự phát triển của công nghệ. Một người dùng điện thoại thông minh giờ đây có thể thực hiện giao dịch tài chính, nghe nhạc, lướt web, xem phim ngay trên thiết bị của mình. Một máy hút bụi thông minh được trang bị các cảm biến nên hoàn toàn có thể sử dụng dữ liệu cảm biến này như một “time series”. Chi phí truyền tin từ các thiết bị biên đến máy chủ để huấn luyện trở nên tốn kém và có thể gây mất thông tin, ảnh hưởng đến hiệu suất học của mô hình.

\textbf{Chi phí phần cứng máy chủ.} Sau khi dữ liệu được gửi về máy chủ, cần một cấu hình máy mạnh mẽ cùng khả năng lưu trữ lớn để có thể xử lý hết lượng dữ liệu khổng lồ trên trong thời gian giới hạn.

Việc các phương pháp tiếp cận máy học theo hướng truyền thống đang dần bộc lộ các nhược điểm về chi phí vận hành và bảo trì ngày càng cao, cũng như các mối nguy hiểm tiềm tàng có thể xảy ra đối với dữ liệu của người dùng, một lần nữa thúc đẩy việc nghiên cứu về một phương pháp huấn luyện giúp làm giảm chi phí phần cứng (sử dụng cho đường truyền và máy chủ), đồng thời đảm bảo tính riêng tư dữ liệu cho người dùng.

Khái niệm \textit{federated learning} và thuật toán \codeword{FedAvg} được đưa ra vào năm 2016 bởi Google trong nghiên cứu \cite{mcmahan2017communication} nhằm mục đích huấn luyện mô hình máy học trên các tập dữ liệu riêng biệt được phân bố trên các thiết bị biên (được gọi là huấn luyện phân tán). Do đó, một hệ thống FL không cần một máy chủ quá mạnh để vận hành (thậm chí có thể sử dụng một máy khách để vận hành \cite{yin2021comprehensive}), không đòi hỏi chi phí truyền tin quá lớn và đảm bảo được quyền riêng tư dữ liệu của người dùng vì không diễn ra bất cứ quá trình thu thập dữ liệu từ người dùng nào (điều mà mô hình huấn luyện tập trung bắt buộc phải làm). Dễ thấy rằng, phần lớn quá trình tính toán được phân tán đến các thiết bị biên. Tuy nhiên, khả năng lưu trữ và tính toán tại các thiết bị này ngày càng được cải thiện, khiến cho việc việc huấn luyện phân tán dần trở nên khả thi và đạt hiệu quả cao hơn.

Bên cạnh đó, nghiên cứu \cite{zhao2018federated} chỉ ra rằng, hệ thống FL hoạt động trên nền thuật toán \codeword{FedAvg} bị giảm hiệu suất nghiêm trọng khi xử lý dữ liệu Non-IID. Tuy nhiên, dữ liệu phân bố trên máy khách là không đồng nhất và có tính cá nhân hóa tất cao. Nói cách khác, các tập dữ liệu này tuân theo phân phối Non-IID. Do đó, \textbf{khóa luận này được thực hiện nhằm nghiên cứu và cải tiến hệ thống FL để thu được kết quả cao cũng như cải thiện khả năng cá nhân hóa mô hình học cho từng người dùng trên dữ liệu Non-IID}.

\section{Phạm vi đề tài}

Nghiên cứu \cite{yin2021comprehensive} chỉ ra ba hướng nghiên cứu chính khi đề cập đến một hệ thống FL: (1) - Cải thiện hiệu suất của hệ thống FL, (2) - Cải thiện khả năng bảo mật của hệ thống FL, (3) - Cải thiện vấn đề về quyền riêng tư của người dùng trong hệ thống FL.

Về việc phân loại hệ thống FL, dựa trên dữ liệu đầu vào, hệ thống FL được chia thành ba loại \cite{yin2021comprehensive}: (1) - Horizontal FL, (2) - Vertical FL, (3) - Federated transfer learning.

Về việc phân loại các kịch bản Non-IID, nghiên cứu \cite{zhu2021federated} chỉ ra bốn kịch bản chính: (1) - Phân phối thuộc tính khác nhau giữa các máy khách, (2) - Phân phối nhãn khác nhau giữa các máy khách, (3) - Phân phối thời gian khác nhau giữa các máy khách, (4) - Các kịch bản khác.

Từ đó, khóa luận này có phạm vi nghiên cứu được giới hạn và phương án giải quyết được xây dựng dựa trên ba giả định sau:

\begin{itemize}
    \item Loại hệ thống: Môi trường thí nghiệm (bao gồm các yếu tố như số lượng người dùng, dữ liệu, cấu hình, khả năng lưu trữ của thiết bị cuối,...) tuân theo đặc trưng của hệ thống Horizontal FL.
    \item Bảo mật \& quyền riêng tư: Hệ thống đã đảm bảo tính bảo mật cũng như duy trì tốt quyền riêng tư của người dùng.
    \item Kịch bản Non-IID: Phân phối nhãn dữ liệu trên các máy khách là khác nhau.
\end{itemize}

\section{Đóng góp chính}

Các đóng góp chính của khóa luận được chia thành hai loại: đóng góp về mặt lý thuyết và đóng góp về mặt thực nghiệm.

\subsection{Đóng góp lý thuyết}

\begin{itemize}
    \item Nghiên cứu hệ thống FL và thách thức về phân phối dữ liệu mà hệ thống Horizontal FL gặp phải.
    \item Khảo sát các phương pháp tối ưu hóa hệ thống Horizontal FL trên dữ liệu dạng Non-IID. Trong đó, tập trung nghiên cứu các phương pháp theo hướng Personalized Federated Averaging \parencite{fallah2020personalized, chen2018federated} và Personalization Layer \parencite{liang2020think, arivazhagan2019federated}.
    \item Phương pháp đề xuất của khóa luận đã cho thấy khả năng đạt được độ chính xác cao hơn trong quá trình kiểm thử với hai đối tượng người dùng (người dùng cục bộ và người dùng mới) so với các phương pháp trước đó (chỉ sử dụng \codeword{FedAvg}, chỉ sử dụng Personalized Federated Averaging, hoặc chỉ sử dụng Personalization Layer).
\end{itemize}

\subsection{Đóng góp thực nghiệm}

\begin{itemize}
    \item Tổ chức bộ dữ liệu MNIST và CIFAR-10 theo hai hướng IID và Non-IID để tiến hành thí nghiệm.
    \item Cài đặt thuật toán \codeword{FedAvg}, \codeword{FedAvgMeta}, các thuật toán kết hợp giữa \codeword{FedAvg} và ML (thuật toán \codeword{FedMeta(MAML)}, \codeword{FedMeta(Meta-SGD)}).
    \item Cài đặt thuật toán kết hợp giữa \codeword{FedAvg} và PL (thuật toán \codeword{FedPer}, \codeword{LG-FedAvg}).
    \item Kết hợp các thuật toán ML và PL vào hệ thống FL: các thuật toán \codeword{FedMeta-Per}.
    \item Fine-tune các siêu tham số như số lượng máy khách tham gia huấn luyện, số bước huấn luyện cục bộ, các siêu tham số học để mô hình đạt độ chính xác tốt nhất.
\end{itemize}

\section{Bố cục}

Trong luận văn này, chương \ref{Chapter2} trình bày về tổng quan lý thuyết được sử dụng trong khóa luận, các lý thuyết này làm nền tảng cho nghiên cứu và đề xuất thuật toán; chương \ref{Chapter3} đề xuất thuật toán giúp giải quyết vấn đề vừa nêu ở chương \ref{Chapter1}; chương \ref{Chapter4} trình bày về cài đặt thực nghiệm để kiểm chứng tính hiệu quả của thuật toán; chương \ref{Chapter5} đi vào phân tích kết quả đạt được; chương \ref{Chapter6} nêu kết luận, những điều chưa làm được và hướng phát triển tương lai của khóa luận.


\chapter{Tổng quan lý thuyết}
\label{Chapter2}

\section{Hệ thống Federated Learning}

\subsection{Định nghĩa}

\textbf{Định nghĩa về FL} \cite{yang2019federated}: Giả sử có $n$ máy khách, máy khách thứ $i$ ký hiệu là $c_i$ $(i\in [1, n])$, chứa tập dữ liệu $D_i$. FL là một quá trình học mà ở đó, các chủ sở hữu dữ liệu (ở đây có thể hiểu là các thiết bị biên) cùng hợp tác huấn luyện một mô hình $\mathcal{M}$ và đạt được độ chính xác $f$ nhưng không có bất kỳ chủ sở hữu dữ liệu $c_i$ nào chia sẻ tập dữ liệu $\mathcal{D}_i$ của chúng.

Gọi $\bar{\mathcal{M}}$ là mô hình máy học được huấn luyện trên tập dữ liệu $\mathcal{D}  = \mathcal{D}_1 \cup \mathcal{D}_2 \cup ... \cup \mathcal{D}_n$ và cho độ chính xác $\bar{f}$. $f$ và $\bar{f}$ chỉ được phép chênh lệch nhau một khoảng nhỏ. Gọi $\delta$ là một giá trị thực không âm, nếu $\mid f-\bar{f}\mid < \delta$ ta nói mô hình $\mathcal{M}$ có \textit{$\delta$ - accuracy loss}.

\textbf{Định nghĩa về tính hợp lệ} \cite{li2021survey}: Ký hiệu $\mathcal{M}_i$ là mô hình được huấn luyện trên tập dữ liệu $\mathcal{D}_i$ và cho độ chính xác $f_i$. Mô hình $\mathcal{M}$ được gọi là hợp lệ nếu tồn tại $i\in [1,n]$ sao cho $f>f_i$.

\subsection{Một hệ thống Federated Learning điển hình}

\textbf{Thành phần và các tương tác trong hệ thống.} Một hệ thống FL (Hình \ref{fig:fl}) thường bao gồm hai thành phần chính: máy chủ (đóng vai trò là đối tượng duy trì mô hình toàn cục) và máy khách (đóng vai trò là đối tượng nắm giữ dữ liệu huấn luyện). Hai thành phần này tương tác với nhau theo ba bước sau \cite{lim2020federated}:

\begin{itemize}
    \item \textit{Khởi tạo.} Máy chủ khởi tạo trọng số $w_G^0$ cho mô hình toàn cục và các siêu tham số cho quá trình huấn luyện. Thông tin này sau đó được gửi đến một tập hợp con các máy khách được chọn để tiến hành huấn luyện.

    \item \textit{Huấn luyện và cập nhật mô hình cục bộ.} Tại bước huấn luyện thứ $t$, máy khách $c_i$ nhận trọng số $w_G^t$ từ máy chủ và tiến hành huấn luyện cục bộ trên tập dữ liệu $\mathcal{D}_i$. Tham số $\theta_i^{t+1}$ thu được sau quá trình huấn luyện (có thể là trọng số $w_i^{t+1}$ hoặc đạo hàm hàm lỗi $g_i^{t+1}$) được máy khách gửi về máy chủ để tổng hợp.

    \item \textit{Tổng hợp và cập nhật mô hình toàn cục.} Máy chủ nhận tham số $\theta_i^{t+1}$ gửi về từ các máy khách được chọn trước đó, tiến hành tổng hợp $w_G^{t+1}$ - trọng số mới của mô hình toàn cục và gửi trọng số này đến một tập hợp con các máy khách khác để bắt đầu bước huấn luyện toàn cục mới.
\end{itemize}

Máy chủ sẽ lặp lại bước 2 và bước 3 cho đến khi độ lỗi hội tụ hoặc độ chính xác đạt đến một ngưỡng nhất định. Khi quá trình huấn luyện kết thúc, tham số của mô hình toàn cục sẽ được phân phối đến toàn bộ máy khách trong hệ thống.

\begin{figure}[H]
    \begin{center}
        \includegraphics[scale=0.85]{images/fl.png}
        \caption{Hai thành phần chính và quá trình tương tác giữa chúng trong hệ thống FL \cite{chandorikar_2020}}
        \label{fig:fl}
    \end{center}
\end{figure}

\label{purpose_fl}
\textbf{Mục tiêu của hệ thống FL.} Tại đây khảo sát hai mục tiêu của hệ thống FL: (1) - Mục tiêu cục bộ; (2) - Mục tiêu toàn cục.

Các máy khách trong hệ thống hướng đến việc thực hiện mục tiêu cục bộ. Ban đầu, máy khách $c_i$ nhận một trọng số toàn cục $w_G$ từ máy chủ. Máy khách này sau đó sẽ cố gắng tìm kiếm một trọng số $w_i^{t+1}$ giúp cực tiểu hóa hàm lỗi cục bộ. Nói cách khác, $w_i^{t+1}$ phải thỏa mãn thỏa mãn:

\begin{equation}
    \label{eq:opt_client}
    w_i^{t+1} = \arg\min_{w_i}{f_{local}(w_i^t)}
\end{equation}

Trong đó, $f_{local}(w_i)$ là hàm lỗi trên tập dữ liệu của $c_i$. Với $\alpha$ là siêu tham số học cục bộ, $w_{i(j)}$ là trọng số tại bước huấn luyện $j$ của $c_i$, lời giải của phương trình \ref{eq:opt_client} theo phương pháp SGD sau $e$ bước huấn luyện có thể được viết như sau:

\begin{equation}
    \begin{cases}
        w_{i(0)}^t = w_G^t\\
        w_{i(j)}^t = w_{i(j-1)}^t - \alpha \nabla f_{local}(w_{i(j-1)}^t)\\
        w_i^{t+1} = w_{i(e)}^t
    \end{cases}
\end{equation}

Hay:

\begin{dmath}
    w_i^{t+1} \leftarrow w_i^t - \alpha\nabla f_{local}(w_i^t)
\end{dmath}

Mặt khác, mục tiêu toàn cục, cũng là mục tiêu chính của hệ thống FL, được máy chủ thực hiện bằng cách tìm kiếm một trọng số $w_G^*$ giúp tối thiểu hóa hàm lỗi của cả hệ thống \cite{yin2021comprehensive}:

\begin{dmath}
    \label{eq:opt_server}
    w_G^* = \arg \min_{w_G}{f_{global}(w_G)}
        = \arg \min_{w_G}{\frac{1}{n} \sum_{i=1}^n{f_{local}(w_i)}}
\end{dmath}

Trong đó, $f_{global}(w_G)$ là hàm lỗi toàn cục của hệ thống. Để giải phương trình \ref{eq:opt_server}, máy chủ thực hiện tổng hợp tham số gửi về từ máy khách bằng một trong hai cách: lấy trung bình trọng số \parencite{mcmahan2017communication, aono2017privacy, yoon2021fedmix} hoặc lấy trung bình đạo hàm \parencite{chen2018federated, mcmahan2017learning}.

Đặt $n_i = |\mathcal{D}_i|$ là số điểm dữ liệu của tập $\mathcal{D}_i$, $N = \sum_{i=1}^n{n_i}$ là tổng số điểm dữ liệu có trong cả hệ thống. Phương pháp lấy trung bình trọng số tính toán trọng số toàn cục tại bước huấn luyện thứ $t$ từ các trọng số của máy khách như sau \cite{mcmahan2017communication}:

\begin{equation}
    \label{eq:agg_w}
    w_G^{t+1} = \sum_{i=1}^n{\frac{n_i}{N} w_i^{t+1}}
\end{equation}

Trái lại, phương pháp lấy trung bình đạo hàm đòi hỏi máy khách gửi về đạo hàm hàm lỗi sau khi kết thúc quá trình huấn luyện cục bộ. Với $\beta$ là siêu tham số học toàn cục, quá trình tổng hợp tại bước huấn luyện $t$ được biểu diễn theo công thức:

\begin{dmath}
    w_G^{t+1} = w_G^t - \beta  \sum_{i=1}^n{\frac{n_i}{N} \nabla f_{local}(w_i^{t+1})}
        = w_G^t - \beta g^{t+1}
\end{dmath}

Sau khi khảo sát cả hai phương pháp tổng hợp tham số của máy chủ, nghiên cứu \cite{yin2021comprehensive} chỉ ra rằng, việc lấy trung bình trọng số giúp hệ thống có khả năng chịu được việc mất cập nhật, nhưng không đảm bảo việc hội tụ. Trái lại, việc lấy trung bình đạo hàm giúp hệ thống đảm bảo sự hội tụ nhưng tiêu tốn nhiều chi phí truyền tin hơn. Để phù hợp hơn với giới hạn về chi phí giao tiếp và lưu trữ, khóa luận này tổng hợp trọng số toàn cục bằng phương pháp lấy trung bình trọng số.

\textbf{Phân loại hệ thống Federated Learning.} Nghiên cứu \cite{yin2021comprehensive} đề xuất các phân loại các hệ thống FL dựa trên phân bố dữ liệu đầu vào của chúng. Theo đó, ba phân bố dữ liệu: (1) - Phân bố dữ liệu theo chiều ngang (Horizontal data partitioning), (2) - Phân bố dữ liệu theo chiều dọc (Vertical data partitioning), (3) - Phân bố dữ liệu hỗn hợp (Hybrid data partitioning) sẽ ứng với ba loại hệ thống FL (Hình \ref{fig:taxonomy_fl}): (1) - Hệ thống FL theo chiều ngang (Horizontal FL), (2) - Hệ thống FL theo chiều dọc (Vertical FL), (3) - Hệ thống học chuyển giao tri thức (Federated Transfer Learning).

\textit{Hệ thống Horizontal FL.} Phân bố dữ liệu theo chiều ngang là kiểu phân bố dữ liệu mà ở đó các bên tham gia vào hệ thống cùng sở hữu các đặc tính dữ liệu giống nhau nhưng giá trị định danh của mẫu dữ liệu của các bên là khác nhau. Ví dụ, khi các bên tham gia hệ thống là các trường đại học, họ sẽ muốn quản lý các thông tin giống nhau về sinh viên như họ và tên, mã số sinh viên,... nhưng không có một sinh viên nào tham gia hai trường đại học cùng một lúc. Kiến trúc Horizontal FL rất phù hợp để huấn luyện mô hình học tuân theo phân phối này \cite{yin2021comprehensive}.

\begin{figure}[H]
    \begin{center}
        \includegraphics[height=20cm]{images/taxonomy_fl.png}
        \caption{Ba loại hệ thống FL với phân bố dữ liệu tương ứng \cite{yang2019federated}}
        \label{fig:taxonomy_fl}
    \end{center}
\end{figure}

Dựa vào kiến trúc giao tiếp, có thể chia Horizontal FL ra làm hai loại: Kiến trúc client-server và kiến trúc peer-to-peer (P2P). Kiến trúc client-server, hay còn gọi là kiến trúc FL tập trung, về cơ bản sẽ thực hiện các bước huấn luyện giống như đã trình bày trong phần \textbf{Thành phần và các tương tác trong hệ thống}. Trong khi đó, kiến trúc P2P, hay còn gọi là kiến trúc FL phân tán không có một máy chủ cố định. Tại mỗi bước huấn luyện toàn cục, một máy khách trong hệ thống được chọn làm máy chủ. Quá trình huấn luyện sau đó được thực hiện giống như kiến trúc client-server.

Một hệ thống Horizontal FL thường có số lượng máy khách rất lớn, khả năng lưu trữ và tính toán tại các máy khách không cần quá cao (ví dụ như điện thoại thông minh, máy tính bảng) và tần suất một máy khách tham gia huấn luyện là rất thấp.

\textit{Hệ thống Vertical FL.} Đây là kiến trúc phù hợp với phân bố dữ liệu theo chiều dọc. Trong phân bố dữ liệu dạng này, các bên tham gia hệ thống sở hữu các đặc tính dữ liệu khác nhau nhưng giá trị định danh của mẫu dữ liệu của các bên là giống nhau. Ví dụ, khi các bên tham gia hệ thống là ngân hàng và trường đại học. Thuộc tính mà ngân hàng và trường đại học lưu trữ là rất khác nhau nhưng có thể chứa thông tin của cùng một người dùng.

\textit{Hệ thống Federated Transfer Learning.} Khi phân bố dữ liệu của các bên tham gia hệ thống không sở hữu chung các đặc tính dữ liệu hay giá trị định danh của từng mẫu, người ta gọi đây là phân bố dữ liệu hỗn hợp. Ví dụ, khi các bên tham gia hệ thống là một ngân hàng ở Hoa Kỳ và một trường đại học ở Việt Nam. Do cản trở địa lý và nhu cầu quản lý thông tin khác nhau, chủ sở hữu dữ liệu này sẽ không có chung thuộc tính hay giá trị định danh nào. Trong trường hợp này, FTL có thể được sử dụng để chuyển giao tri thức giữa các bên tham gia.

Dựa vào các đặc điểm phân loại nêu trên, \textbf{hệ thống FL trong khóa luận này được xếp vào nhóm hệ thống Horizontal FL tập trung, bao gồm một máy chủ quản lý nhiều máy khách}.

\section{Khảo sát dữ liệu Non-IID}

Dữ liệu tại các máy khách thường được sinh ra dựa trên nhu cầu của người dùng cuối. Do đó, loại dữ liệu này thường có tính cá nhân hóa cao và không đồng nhất. Thuật ngữ sử dụng để chỉ phân phối dữ liệu trong các tập dữ liệu này là \textit{dữ liệu Non-IID}. \textbf{Khi nhắc đến dữ liệu Non-IID, người ta ngầm hiểu rằng, không có bất kỳ phân phối dữ liệu cục bộ nào có thể đại diện cho phân phối trên toàn bộ dữ liệu, phân phối dữ liệu trên hai máy khách khác nhau là hoàn toàn khác nhau} \cite{zhu2021federated}.

Về mặt công thức, gọi $(x, y)$ là cặp thuộc tính và nhãn dữ liệu. Phân phối dữ liệu của hai máy khách $c_i, c_j$ bất kỳ được ký hiệu là $p_i(x,y), p_j(x,y)$. Trong kịch bản dữ liệu IID, ta có $p_i(x,y) = p_j(x,y)$. Trái ngược với dữ liệu IID, dữ liệu Non-IID có $p_i(x,y) \ne p_j(x,y)$ (Hình \ref{fig:iid_noniid}).

\begin{figure}[H]
    \centering
    \includegraphics[scale=0.7]{../images/iid_vs_noniid.png}
    \caption{Minh họa kịch bản dữ liệu IID và Non-IID trên tập dữ liệu MNIST \cite{hellstrom2020wireless}}
    \label{fig:iid_noniid}
\end{figure}

% Mặt khác, nghiên cứu \cite{zhao2018federated} chỉ ra rằng \textbf{hệ thống FL có thể bị giảm hiệu quả nghiêm trọng khi làm việc trên dữ liệu Non-IID vì mô hình toàn cục không được huấn luyện trên các tập dữ liệu tuân theo phân phối đều}. 

Mặt khác, nghiên cứu \cite{zhao2018federated} chỉ ra rằng \textbf{hệ thống FL có thể bị giảm hiệu quả nghiêm trọng khi làm việc trên dữ liệu Non-IID. Nguyên nhân được chỉ ra là do đạo hàm trên từng lô (batch) dữ liệu không đại diện mô phỏng được đạo hàm trên toàn bộ dữ liệu}. Để hiểu rõ vấn đề mình đang đối mặt, dựa trên nghiên cứu \cite{zhu2021federated}, khóa luận tiến hành khảo sát bốn kịch bản về dữ liệu Non-IID: (1) - Phân phối thuộc tính khác nhau giữa các máy khách, (2) - Phân phối nhãn khác nhau giữa các máy khách, (3) - Phân phối thời gian khác nhau giữa các máy khách, (4) - Các kịch bản khác.

\subsection{Phân phối thuộc tính khác nhau giữa các máy khách}

Với kịch bản này, phân phối thuộc tính $p(x)$ trên các máy khách là đôi một khác nhau. Không gian thuộc tính của các máy khách có thể khác nhau hoàn toàn, trùng lặp một vài thuộc tính hoặc trùng lặp hoàn toàn.

Các hệ thống Vertical FL thường rơi vào trường hợp đầu tiên. Ví dụ, trong hình \ref{fig:vertical_fl}, các thuộc tính mà máy khách 1 quản lý (\textit{Age, Height}) khác hoàn toàn với các thuộc tính tại máy khách 2 (\textit{Sex, Weight}) nhưng các thuộc tính này là của chung dữ liệu định danh (\textit{Person A, B, C,...}).

\begin{figure}[H]
    \centering
    \includegraphics[scale=0.6]{../images/vertical_fl.png}
    \caption{Minh họa dữ liệu Non-IID trên thuộc tính \cite{zhu2021federated}}
    \label{fig:vertical_fl}
\end{figure}

Đối với trường hợp thứ hai, hai máy khách có thể cùng quản lý một số thuộc tính dữ liệu. Ví dụ, đối với dữ liệu của một hệ thống camera dám sát, hai camera bất kỳ có thể cùng lưu hình một người với các góc chụp khác nhau.

Trường hợp cuối chính là đặc điểm chính của hệ thống Horizontal FL. Tại đây, không gian thuộc tính của các máy khách là hoàn toàn giống nhau. Trong hình \ref{fig:vertical_fl}, nếu máy khách 1 và máy khách cùng quản lý các thuộc tính giống nhau và các mẫu dữ liệu chứa dữ liệu rất khác nhau (ví dụ máy khách 1 chứa tập người dùng trẻ tuổi, máy khách 2 chứa tập người dùng cao tuổi), thì đây là một ví dụ đơn giản cho trường hợp Non-IID này.

\subsection{Phân phối nhãn khác nhau giữa các máy khách}

Đây là trường hợp dữ liệu Non-IID phổ biến nhất, gây hại nghiêm trọng cho hệ thống FL \cite{zhu2021federated}, cũng chính là trường hợp mà khóa luận hướng tới giải quyết. Tại đây, phân phối nhãn của hai máy khách $c_i, c_j$ bất kỳ là khác nhau: $p_i(y) \ne p_j(y)$ và xác suất thuộc tính $x$ có nhãn dữ liệu $y$: $p(x|y)$ của các máy khách là như nhau. Một kịch bản thường thấy của trường hợp này được trình bày trong nghiên cứu \cite{mcmahan2017communication}: Mỗi máy khách sẽ chỉ chứa các điểm dữ liệu thuộc về $k$ nhãn (hình \ref{fig:my_noniid}). Trong đó, $k$ là một siêu tham số biểu diễn độ mất cân bằng về nhãn. $k$ càng nhỏ, hệ thống mất cân bằng nhãn càng mạnh. Khóa luận này thiết đặt môi trường dữ liệu theo cách tương tự như vậy.

\begin{figure}[H]
    \centering
    \includegraphics[scale=0.7]{../images/noniid_label.png}
    \caption{Minh họa dữ liệu Non-IID trên nhãn với của hai máy khách với $k=2$}
    \label{fig:my_noniid}
\end{figure}

Ngoài ra, còn một trường hợp phổ biến khác liên quan đến việc dữ liệu Non-IID trên nhãn. Đối với trường họp này, xác suất thuộc tính $x$ được gán nhãn $y$: $p(y|x)$ là khác nhau giữa các máy khách. Ví dụ, với một bức ảnh trên mạng xã hội, người dùng $A$ có thể gán nhãn "yêu thích", trong khi người dùng $B$ gán nhãn "không yêu thích".

\subsection{Phân phối thời gian khác nhau giữa các máy khách}

Một ví dụ dễ hiệu cho trường hợp này là việc hai người dùng $c_i$, $c_j$ thu thập dữ liệu trong hai khoảng thời gian khác nhau. Dẫn đến việc phân phối $p_i(x, y|t) \ne p_j(x, y|t)$, với $t$ là một thời điểm nhất định. Một trường hợp khác của kịch bản này là phân phối $p(x, y|t)$ của một máy khách bị thay đổi liên tục theo thời gian. Ví dụ, một hệ thống camera giám sát có thể ghi nhận hình ảnh của rất nhiều người vào các ngày làm việc trong tuần nhưng lại có rất ít hình ảnh vào những ngày nghỉ.

\subsection{Các kịch bản khác}

Các kịch bản còn lại thường rơi vào hai trường hợp: (1) - Phân phối thuộc tính và nhãn là khác nhau giữa các máy khách, (2) - Số lượng dữ liệu huấn luyện là khác nhau giữa các máy khách.

\section{Tối ưu hệ thống Federated Learning trên dữ liệu Non-IID}

\subsection{Tối ưu dựa trên dữ liệu}

Việc mô hình toàn cục không được làm việc với dữ liệu tuân theo phân phối đều trên từng người dùng, mà thay vào đó là phân phối Non-IID khiến cho hiệu suất hệ thống bị sụt giảm nghiêm trọng \cite{zhao2018federated}. Hướng tối ưu dựa trên dữ liệu trực tiếp giải quyết vấn đề này bằng hai cách: (1) - Chia sẻ dữ liệu, (2) - Tăng cường dữ liệu.

\textbf{Chia sẻ dữ liệu.} Chia sẻ dữ liệu \cite{zhu2021federated} được thực hiện bằng cách xây dựng một tập dữ liệu chứa dữ liệu của tất cả các nhãn theo phân phối đều. Dữ liệu trong tập này được thu thập trực tiếp từ các máy khách và gửi về máy chủ để kết hợp huấn luyện mô hình toàn cục.

\textbf{Tăng cường dữ liệu.} Cùng với ý tưởng cho phép mô hình toàn cục được học trên các tập dữ liệu có phân phối đều các nhãn trong hệ thống, tăng cường dữ liệu \cite{tanner1987calculation} nhằm mục đích gia tăng sự đa dạng của dữ liệu huấn luyện. Phương pháp này đòi hỏi các máy khách phải gửi phân phối dữ liệu của mình về máy chủ. Máy chủ theo đó yêu cầu các máy khách tạo ra ảnh mới \cite{duan2019astraea} với số lượng và nhãn lớp biết trước, hoặc tự mình tạo ra ảnh mới bằng cách sử dụng GAN \cite{zhu2021federated} để có thể huấn luyện trên một tập dữ liệu chứa tất cả các nhãn với phân phối trung bình.

Các phương pháp nêu trên đều giúp hệ thống FL "chống chịu" tốt trước dữ liệu Non-IID. Tuy nhiên, đỏi hỏi máy khách gửi thông tin cá nhân về máy chủ là vi phạm mục tiêu ban đầu của hệ thống FL - bảo vệ quyền riêng tư dữ liệu của người dùng.

\subsection{Tối ưu dựa trên thuật toán}

\subsubsection{Fine-tune cục bộ}

Thực hiện fine-tune, hay tinh chỉnh mô hình ở máy khách, là kỹ thuật mạnh mẽ trong việc cá nhân hóa mô hình học cho các tập dữ liệu riêng biệt. Kỹ thuật này hướng đến việc fine-tune mô hình học tại các máy khách sau khi nhận được mô hình từ máy chủ \cite{wang2019federated}.

Một hướng tiếp cận phổ biến được đề ra là sử dụng ML trong việc tạo ra một mô hình toàn cục tốt, có thể thích ứng với tập dữ liệu mới trên máy khách một cách nhanh chóng. Các thuật toán theo hướng này \parencite{chen2018federated, fallah2020personalized} sử dụng các kỹ thuật ML có khả năng tạo ra một khởi tạo tốt như \codeword{Model-Agnostic Meta-Learning (MAML)} \cite{finn2017model} hay \codeword{Meta-SGD} \cite{li2017meta} để huấn luyện mô hình toàn cục. Mô hình toàn cục này, trong quá trình chạy thực tế trên một máy khách mới, hoàn toàn có thể đạt hội tụ chỉ sau một hoặc một vài bước huấn luyện.

\subsubsection{Lớp cá nhân hóa}

Các thuật toán theo hướng này cho phép duy trì một phần của mạng học sâu trên máy khách. Cụ thể, thuật toán chia mạng học sâu thành hai thành phần: phần chung và phần riêng. Phần chung được hợp tác huấn luyện bởi các máy khách và được tổng hợp bởi máy chủ. Phần riêng tồn tại riêng biệt trên từng máy khách, được máy khách trực tiếp duy trì và huấn luyện.

Một thuật toán điển hình theo hướng tiếp cận này là \codeword{FedPer} \cite{arivazhagan2019federated}. \codeword{FedPer} quy định phần chung của mạng học sâu là các lớp rút trích đặc trưng, phần riêng của mạng là các lớp còn lại. Tác giả của \codeword{FedPer} đã thực hiện nhiều thí nghiệm chứng tỏ rằng thuật toán này đạt điệu quả cao hơn nhiều so với \codeword{FedAvg} khi làm việc trên dữ liệu Non-IID.

Thuật toán \codeword{LG-FedAvg} \cite{liang2020think} cũng được xếp vào nhóm thuật toán sử dụng lớp cá nhân hóa. Tuy nhiên, ngược lại với \codeword{FedPer}, \codeword{LG-FedAvg} chỉ định phần riêng là các lớp rút trích đặc trưng trong mạng học sâu. Thực nghiệm cho thấy, \codeword{LG-FedAvg} đạt hiệu quả tốt hơn \codeword{FedAvg} trên cả các máy khách sẵn có trong hệ thống lẫn các máy khách chỉ vừa mới tham gia hệ thống.

\label{open_question}
Bằng các lớp cá nhân hóa, các thuật toán nêu trên đã giải quyết được sự khác nhau về dữ liệu giữa các máy khách, từ đó tránh được phần nào sự giảm hiệu suất trên dữ liệu Non-IID. Nhưng các lớp thuộc phần chung của mạng học sâu được huấn luyện theo thuật toán \codeword{FedAvg} nên vẫn có thể bị giảm hiệu suất rất lớn khi dữ liệu huấn luyện bị Non-IID nặng. \textbf{Vậy có thể làm gì để các lớp này có thể thích ứng nhanh với tập dữ liệu của máy khách chỉ sau một vài bước huấn luyện?}

\subsection{Tối ưu dựa trên hệ thống}

\subsubsection{Gom cụm người dùng}

Cách tiếp cận FL truyền thống giả định rằng hệ thống này chỉ bao gồm một máy chủ. Điều này làm cho việc học các đặc tính của tất cả các máy khách trong môi trường dữ liệu Non-IID là khó khả thi. Để giải quyết vấn đề này, một hệ thống huấn luyện với nhiều máy chủ được đề xuất. Câu hỏi đặt ra là: Làm sao để biết một máy khách nên huấn luyện cùng với máy chủ nào? 

Trong ngữ cảnh đa máy chủ, thuật toán \codeword{IFCA} \cite{ghosh2020efficient} trả lời câu hỏi trên bằng cách gửi trọng số của tất cả các máy chủ cho từng máy khách. Các máy khách theo đó tìm ra được trọng số cho độ lỗi nhỏ nhất trên tập dữ liệu cục bộ và gửi thông tin sau khi huấn luyện cục bộ của mình về máy chủ đó để cập nhật mô hình toàn cục. Việc gửi toàn bộ trọng số của các máy chủ đến một máy khách khiến cho thuật toán này tăng chi phí giao tiếp lên gấp $k$ lần, với $k$ là số lượng máy chủ của hệ thống.

Một cách khác để trả lời câu hỏi trên là đánh giá sự tương đồng của trọng số do máy khách gửi về \cite{zhu2021federated}. Một thang độ đo sự tương đồng như độ đo cosine được máy chủ sử dụng trên các trọng số của máy khách. Từ đó biết được nên tổng hợp trọng số của các máy khách nào với nhau.

Với việc lượng dữ liệu và thiết bị biên ngày càng tăng lên, việc duy trì nhiều máy chủ là thực sự cần thiết khi đối mặt với nhu cầu nâng cấp hệ thống học. Tuy nhiên, chi phí giao tiếp và phương pháp gom cụm vẫn là những vấn đề rất lớn cần giải quyết.

% \subsection{Meta Learning}

% \subsection{Personalization Layer}


\chapter{Phương pháp đề xuất}
\label{Chapter3}


% Công trình của tác giả (nếu không có thì comment 02 dòng dưới)
\addcontentsline{toc}{chapter}{Danh mục công trình của tác giả}
\chapter*{Danh mục công trình của tác giả}
\label{Appendix1}

\begin{enumerate}
\item Tạp chí ABC
\item Tạp chí XYZ
\end{enumerate}

% In tài liệu tham khảo
\addcontentsline{toc}{chapter}{Tài liệu tham khảo}
\printbibheading[title={Tài liệu tham khảo}]

\printbibliography[heading=subbibliography, title={Tiếng Việt}, keyword=Viet, resetnumbers=true]

\DeclareNameAlias{sortname}{last-first}
\DeclareNameAlias{default}{last-first}

\printbibliography[heading=subbibliography, title={Tiếng Anh}, notkeyword=Viet, resetnumbers=4] 
% ===================================================================== %
% CHÚ Ý: phải gán lại resetnumbers=số tài liệu tham khảo tiếng Việt + 1 %
% ===================================================================== %

% Phần phụ lục
\appendix

\chapter{Ngữ pháp tiếng Việt}
\label{Appendix1}

Đây là phụ lục.
\chapter{Ngữ pháp tiếng Nôm}
\label{Appendix2}

Đây là phụ lục 2.

\end{document} 
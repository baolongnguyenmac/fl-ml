\chapter{Kết luận}
\label{Chapter6}

Các tập dữ liệu không đồng nhất và có tính cá nhân hóa cao được phân bố trên các thiết bị biên của người dùng cuối đòi hỏi một hệ thống FL hiện nay cần có khả năng làm việc trên dữ liệu Non-IID. Đứng trước vấn đề này, bằng các kết hợp các thuật toán ML (thuật toán \codeword{MAML}, \codeword{Meta-SGD}) và các kỹ thuật PL (thuật toán \codeword{FedPer}, \codeword{LG-FedAvg}) vào hệ thống FL, khóa luận đề xuất thuật toán \codeword{FedMeta-Per}, một giải pháp giúp làm tăng độ chính xác lẫn tính cá nhân hóa trên từng người dùng. Trong đó, các lớp phần chung của mạng mạng học sâu được huấn luyện bằng các thuật toán ML giúp hệ thống thích ứng nhanh trên dữ liệu mới, các lớp phần riêng được duy trì tại thiết bị biên giúp làm tăng tính cá nhân hóa của mô hình học trên dữ liệu cục bộ.

Bằng thực nghiệm, khóa luận đã kiểm tra được tính hiểu quả của thuật toán đề xuất trên cả hai phương diện tăng độ chính xác và tăng tính cá nhân hóa của thuật toán đề xuất trên 50 người dùng lần lượt chứa dữ liệu của hai tập dữ liệu CIFAR-10 và MNIST so với các thuật toán có sự kết hợp của FL và ML (\codeword{FedMeta(MAML)}, \codeword{FedMeta(Meta-SGD)}) và thuật toán sử dụng kỹ thuật PL (\codeword{FedPer}). Trong đó, có thể giải thích việc đạt được kết quả cao dựa vào hai yếu tố mang tính thừa kế: (1) - Khả năng thích ứng nhanh trên tập dữ liệu mới của thuật toán đề xuất thừa hưởng từ các thuật toán ML, (2) - Khả năng cá nhân hóa cao cho từng người dùng kế thừa từ các lớp cá nhân hóa của PL và việc fine-tune dữ liệu của ML.

\textbf{Định hướng phát triển.} Thuật toán mà khóa luận đề xuất không chỉ dừng lại ở việc kết hợp bốn thuật toán kể trên mà còn có khả năng phát triển thêm dựa theo hai hướng đi lớn: (1) - Sự kết hợp các thuật toán ML theo hướng tối ưu hai cấp độ vào hệ thống FL, (2) - Việc tìm kiếm và phân cụm người dùng sao cho mỗi người dùng tìm được bộ tham số huấn luyện tốt nhất. Đối với hướng đi đầu tiên, các thuật toán ML như \codeword{FO-MAML} \cite{finn2017model}, \codeword{Reptile} \cite{nichol2018first}, \codeword{iMAML} \cite{rajeswaran2019meta} hoàn toàn có thể được tích hợp vào hệ thống. Đối với hướng đi thứ hai, cần tìm ra một độ đo tốt để việc phân cụm người dùng đạt hiệu quả cao hơn trên cả kết quả phân cụm lẫn chi phí tính toán phải bỏ ra.

Các nghiên cứu trình bày trong khóa luận chỉ thiên về hướng cải thiện độ chính xác của hệ thống. Trong khi đó, cải thiện về phần cứng cũng như vấn đề quyền riêng tư chưa được xét đến. Đây cũng là một trong những hướng đi quan trọng để nâng cao hiệu quả của hệ thống trong tương lai và cần được nghiên cứu nhiều hơn.

Cuối cùng, khóa luận này đóng góp một phần nhỏ vào việc khảo sát ưu, nhược điểm của các phương pháp tối ưu hệ thống hiện tại và làm động lực cho việc kết hợp những ưu điểm của chúng vào cùng một hệ thống để đạt được hiệu quả tốt hơn.

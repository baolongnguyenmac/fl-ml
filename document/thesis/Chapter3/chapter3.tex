\chapter{Hướng dẫn sử dụng template}
\label{Chapter3}

\section{Trích dẫn tài liệu}

Dùng lệnh $\textbackslash cite$ để trích dẫn một hoặc nhiều tài liệu tham khảo.
Tài liệu tham khảo có thể là trang web~\cite{Listings,HDLVThS}, bài báo khoa học~\cite{1994-Cavnar}, sách~\cite{1984-TeX-Knuth,2006-DDien,2006-NPTV}, bài tạp chí~\cite{1989-TED} hoặc các nguồn tham khảo khác. 
Lưu ý khi trích dẫn tài liệu tham khảo, cần viết câu sao cho bỏ phần trong cặp ngoặc vuông đi thì câu vẫn đầy đủ ý nghĩa.
Ví dụ, thay vì viết ``Nghiên cứu~\cite{1989-TED} chỉ ra rằng ... '' thì nên viết ``Nghiên cứu của Zhang~\cite{1989-TED} chỉ ra rằng ...''.
Một ví dụ khác, thay vì viết ``... như trong công trình nghiên cứu~\cite{1994-Cavnar}.'' thì nên viết ``... như trong công trình nghiên cứu của Cavnar và Trenkle~\cite{1994-Cavnar}.''

\section{Chèn mã nguồn}

Để chèn mã nguồn, cần dùng package listings~\cite{Listings}:

\begin{lstlisting}
\usepackage{listings}
\end{lstlisting}

Mã nguồn có thể được chèn trực tiếp như sau:

\begin{lstlisting}[language=Python]
print "Hello, World!"
\end{lstlisting}

hoặc chèn thông qua tập tin chứa mã nguồn trong thư mục $SourceCode$ như sau:

\lstinputlisting[language=C++]{SourceCode/hello.cpp}

Để chèn mã giả, cần dùng package algorithm~\cite{Algorithm}:

\begin{lstlisting}
\usepackage{algorithm}
\end{lstlisting}

Có thể chèn mã giả vào như sau:

\begin{algorithm}
\caption{My algorithm}\label{euclid}
\begin{algorithmic}[1]
\Procedure{MyProcedure}{}
\State $\textit{stringlen} \gets \text{length of }\textit{string}$
\State $i \gets \textit{patlen}$
\BState \emph{top}:
\If {$i > \textit{stringlen}$} \Return false
\EndIf
\State $j \gets \textit{patlen}$
\BState \emph{loop}:
\If {$\textit{string}(i) = \textit{path}(j)$}
\State $j \gets j-1$.
\State $i \gets i-1$.
\State \textbf{goto} \emph{loop}.
\State \textbf{close};
\EndIf
\State $i \gets i+\max(\textit{delta}_1(\textit{string}(i)),\textit{delta}_2(j))$.
\State \textbf{goto} \emph{top}.
\EndProcedure
\end{algorithmic}
\end{algorithm}

\section{Hình ảnh}

Để chèn hình ảnh, cần dùng package graphicx~\cite{Figures}:

\begin{lstlisting}
\usepackage{graphicx}
\end{lstlisting}

Hình \ref{fig:vd1}, hình \ref{fig:vd2} là một số ví dụ về chèn hình ảnh.

\begin{figure}[htp]
\centering
\includegraphics[width=6 cm]{images/logo-khtn.png}
\caption{Hình ví dụ 1}
\label{fig:vd1}
\end{figure}

\begin{figure*}[htp]
\centering
\includegraphics[width=40 mm]{images/logo-khtn.png}
\caption{Hình ví dụ 2}
\label{fig:vd2}
\end{figure*}

\section{Bảng biểu}

Để tạo bảng biểu, tham khảo thêm tại sharelatex.com~\cite{Tables}.
Bảng~\ref{tab:vd1} là một ví dụ về bảng.
Ngoài ra, có một số tool online~\footnote{Như trang http://www.tablesgenerator.com/} có thể được dùng để tạo bảng biểu một cách trực quan.

\begin{table}[ht]
\caption{Bảng ví dụ 1}

\label{tab:vd1}%
\begin{center}
\begin{tabular}{ |p{3cm}||p{3cm}|p{3cm}|p{3cm}|  }
 \hline
 \multicolumn{4}{|c|}{Country List} \\
 \hline
 Country Name     or Area Name& ISO ALPHA 2 Code &ISO ALPHA 3 Code&ISO numeric Code\\
 \hline
 Afghanistan   & AF    &AFG&   004\\
 Aland Islands&   AX  & ALA   &248\\
 Albania &AL & ALB&  008\\
 Algeria    &DZ & DZA&  012\\
 American Samoa&   AS  & ASM&016\\
 Andorra& AD  & AND   &020\\
 Angola& AO  & AGO&024\\
 \hline
\end{tabular}
\end{center}
\end{table}

\section{Công thức}

Công thức có thể chèn vào trong cùng một dòng như $ \sqrt{a^2+b^2} $ hoặc nằm trên dòng riêng như công thức~\ref{eu_eqn}.

\begin{equation} \label{eu_eqn}
x = a_0 + \frac{1}{a_1 + \frac{1}{a_2 + \frac{1}{a_3 + a_4}}}
\end{equation}